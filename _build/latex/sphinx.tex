% Generated by Sphinx.
\def\sphinxdocclass{report}
\documentclass[letterpaper,10pt,english]{sphinxmanual}
\usepackage[utf8]{inputenc}
\DeclareUnicodeCharacter{00A0}{\nobreakspace}
\usepackage{cmap}
\usepackage[T1]{fontenc}
\usepackage{babel}
\usepackage{times}
\usepackage[Bjarne]{fncychap}
\usepackage{longtable}
\usepackage{sphinx}
\usepackage{multirow}

\usepackage{fontspec}     % 引入 fontsepc,這樣才可以用下面的設定字型的指令
\setmainfont{AR PL UKai TW}  % 預設字型
\setsansfont{AR PL UKai TW}  % sans family 字型
\setromanfont{AR PL UKai TW} % roman 字型
\setmonofont{AR PL UKai TW}  % 等寬字型

\XeTeXlinebreaklocale "zh"          % 設定斷行演算法為中文
\XeTeXlinebreakskip = 0pt plus 1pt  % 設定中文字距與英文字距


\title{系統安裝記錄 Documentation}
\date{July 07, 2014}
\release{1.0}
\author{陳國華}
\newcommand{\sphinxlogo}{}
\renewcommand{\releasename}{Release}
\makeindex

\makeatletter
\def\PYG@reset{\let\PYG@it=\relax \let\PYG@bf=\relax%
    \let\PYG@ul=\relax \let\PYG@tc=\relax%
    \let\PYG@bc=\relax \let\PYG@ff=\relax}
\def\PYG@tok#1{\csname PYG@tok@#1\endcsname}
\def\PYG@toks#1+{\ifx\relax#1\empty\else%
    \PYG@tok{#1}\expandafter\PYG@toks\fi}
\def\PYG@do#1{\PYG@bc{\PYG@tc{\PYG@ul{%
    \PYG@it{\PYG@bf{\PYG@ff{#1}}}}}}}
\def\PYG#1#2{\PYG@reset\PYG@toks#1+\relax+\PYG@do{#2}}

\expandafter\def\csname PYG@tok@gh\endcsname{\let\PYG@bf=\textbf\def\PYG@tc##1{\textcolor[rgb]{0.00,0.00,0.50}{##1}}}
\expandafter\def\csname PYG@tok@gr\endcsname{\def\PYG@tc##1{\textcolor[rgb]{1.00,0.00,0.00}{##1}}}
\expandafter\def\csname PYG@tok@c1\endcsname{\let\PYG@it=\textit\def\PYG@tc##1{\textcolor[rgb]{0.25,0.50,0.56}{##1}}}
\expandafter\def\csname PYG@tok@nt\endcsname{\let\PYG@bf=\textbf\def\PYG@tc##1{\textcolor[rgb]{0.02,0.16,0.45}{##1}}}
\expandafter\def\csname PYG@tok@mo\endcsname{\def\PYG@tc##1{\textcolor[rgb]{0.13,0.50,0.31}{##1}}}
\expandafter\def\csname PYG@tok@gp\endcsname{\let\PYG@bf=\textbf\def\PYG@tc##1{\textcolor[rgb]{0.78,0.36,0.04}{##1}}}
\expandafter\def\csname PYG@tok@kd\endcsname{\let\PYG@bf=\textbf\def\PYG@tc##1{\textcolor[rgb]{0.00,0.44,0.13}{##1}}}
\expandafter\def\csname PYG@tok@kn\endcsname{\let\PYG@bf=\textbf\def\PYG@tc##1{\textcolor[rgb]{0.00,0.44,0.13}{##1}}}
\expandafter\def\csname PYG@tok@s2\endcsname{\def\PYG@tc##1{\textcolor[rgb]{0.25,0.44,0.63}{##1}}}
\expandafter\def\csname PYG@tok@vc\endcsname{\def\PYG@tc##1{\textcolor[rgb]{0.73,0.38,0.84}{##1}}}
\expandafter\def\csname PYG@tok@s1\endcsname{\def\PYG@tc##1{\textcolor[rgb]{0.25,0.44,0.63}{##1}}}
\expandafter\def\csname PYG@tok@nn\endcsname{\let\PYG@bf=\textbf\def\PYG@tc##1{\textcolor[rgb]{0.05,0.52,0.71}{##1}}}
\expandafter\def\csname PYG@tok@il\endcsname{\def\PYG@tc##1{\textcolor[rgb]{0.13,0.50,0.31}{##1}}}
\expandafter\def\csname PYG@tok@c\endcsname{\let\PYG@it=\textit\def\PYG@tc##1{\textcolor[rgb]{0.25,0.50,0.56}{##1}}}
\expandafter\def\csname PYG@tok@sc\endcsname{\def\PYG@tc##1{\textcolor[rgb]{0.25,0.44,0.63}{##1}}}
\expandafter\def\csname PYG@tok@gd\endcsname{\def\PYG@tc##1{\textcolor[rgb]{0.63,0.00,0.00}{##1}}}
\expandafter\def\csname PYG@tok@nv\endcsname{\def\PYG@tc##1{\textcolor[rgb]{0.73,0.38,0.84}{##1}}}
\expandafter\def\csname PYG@tok@ne\endcsname{\def\PYG@tc##1{\textcolor[rgb]{0.00,0.44,0.13}{##1}}}
\expandafter\def\csname PYG@tok@cp\endcsname{\def\PYG@tc##1{\textcolor[rgb]{0.00,0.44,0.13}{##1}}}
\expandafter\def\csname PYG@tok@s\endcsname{\def\PYG@tc##1{\textcolor[rgb]{0.25,0.44,0.63}{##1}}}
\expandafter\def\csname PYG@tok@sr\endcsname{\def\PYG@tc##1{\textcolor[rgb]{0.14,0.33,0.53}{##1}}}
\expandafter\def\csname PYG@tok@sd\endcsname{\let\PYG@it=\textit\def\PYG@tc##1{\textcolor[rgb]{0.25,0.44,0.63}{##1}}}
\expandafter\def\csname PYG@tok@go\endcsname{\def\PYG@tc##1{\textcolor[rgb]{0.20,0.20,0.20}{##1}}}
\expandafter\def\csname PYG@tok@gu\endcsname{\let\PYG@bf=\textbf\def\PYG@tc##1{\textcolor[rgb]{0.50,0.00,0.50}{##1}}}
\expandafter\def\csname PYG@tok@kc\endcsname{\let\PYG@bf=\textbf\def\PYG@tc##1{\textcolor[rgb]{0.00,0.44,0.13}{##1}}}
\expandafter\def\csname PYG@tok@ge\endcsname{\let\PYG@it=\textit}
\expandafter\def\csname PYG@tok@gt\endcsname{\def\PYG@tc##1{\textcolor[rgb]{0.00,0.27,0.87}{##1}}}
\expandafter\def\csname PYG@tok@bp\endcsname{\def\PYG@tc##1{\textcolor[rgb]{0.00,0.44,0.13}{##1}}}
\expandafter\def\csname PYG@tok@kt\endcsname{\def\PYG@tc##1{\textcolor[rgb]{0.56,0.13,0.00}{##1}}}
\expandafter\def\csname PYG@tok@sx\endcsname{\def\PYG@tc##1{\textcolor[rgb]{0.78,0.36,0.04}{##1}}}
\expandafter\def\csname PYG@tok@se\endcsname{\let\PYG@bf=\textbf\def\PYG@tc##1{\textcolor[rgb]{0.25,0.44,0.63}{##1}}}
\expandafter\def\csname PYG@tok@cm\endcsname{\let\PYG@it=\textit\def\PYG@tc##1{\textcolor[rgb]{0.25,0.50,0.56}{##1}}}
\expandafter\def\csname PYG@tok@si\endcsname{\let\PYG@it=\textit\def\PYG@tc##1{\textcolor[rgb]{0.44,0.63,0.82}{##1}}}
\expandafter\def\csname PYG@tok@m\endcsname{\def\PYG@tc##1{\textcolor[rgb]{0.13,0.50,0.31}{##1}}}
\expandafter\def\csname PYG@tok@o\endcsname{\def\PYG@tc##1{\textcolor[rgb]{0.40,0.40,0.40}{##1}}}
\expandafter\def\csname PYG@tok@sh\endcsname{\def\PYG@tc##1{\textcolor[rgb]{0.25,0.44,0.63}{##1}}}
\expandafter\def\csname PYG@tok@vi\endcsname{\def\PYG@tc##1{\textcolor[rgb]{0.73,0.38,0.84}{##1}}}
\expandafter\def\csname PYG@tok@mi\endcsname{\def\PYG@tc##1{\textcolor[rgb]{0.13,0.50,0.31}{##1}}}
\expandafter\def\csname PYG@tok@kr\endcsname{\let\PYG@bf=\textbf\def\PYG@tc##1{\textcolor[rgb]{0.00,0.44,0.13}{##1}}}
\expandafter\def\csname PYG@tok@nf\endcsname{\def\PYG@tc##1{\textcolor[rgb]{0.02,0.16,0.49}{##1}}}
\expandafter\def\csname PYG@tok@err\endcsname{\def\PYG@bc##1{\setlength{\fboxsep}{0pt}\fcolorbox[rgb]{1.00,0.00,0.00}{1,1,1}{\strut ##1}}}
\expandafter\def\csname PYG@tok@ow\endcsname{\let\PYG@bf=\textbf\def\PYG@tc##1{\textcolor[rgb]{0.00,0.44,0.13}{##1}}}
\expandafter\def\csname PYG@tok@nb\endcsname{\def\PYG@tc##1{\textcolor[rgb]{0.00,0.44,0.13}{##1}}}
\expandafter\def\csname PYG@tok@na\endcsname{\def\PYG@tc##1{\textcolor[rgb]{0.25,0.44,0.63}{##1}}}
\expandafter\def\csname PYG@tok@no\endcsname{\def\PYG@tc##1{\textcolor[rgb]{0.38,0.68,0.84}{##1}}}
\expandafter\def\csname PYG@tok@vg\endcsname{\def\PYG@tc##1{\textcolor[rgb]{0.73,0.38,0.84}{##1}}}
\expandafter\def\csname PYG@tok@cs\endcsname{\def\PYG@tc##1{\textcolor[rgb]{0.25,0.50,0.56}{##1}}\def\PYG@bc##1{\setlength{\fboxsep}{0pt}\colorbox[rgb]{1.00,0.94,0.94}{\strut ##1}}}
\expandafter\def\csname PYG@tok@sb\endcsname{\def\PYG@tc##1{\textcolor[rgb]{0.25,0.44,0.63}{##1}}}
\expandafter\def\csname PYG@tok@k\endcsname{\let\PYG@bf=\textbf\def\PYG@tc##1{\textcolor[rgb]{0.00,0.44,0.13}{##1}}}
\expandafter\def\csname PYG@tok@nd\endcsname{\let\PYG@bf=\textbf\def\PYG@tc##1{\textcolor[rgb]{0.33,0.33,0.33}{##1}}}
\expandafter\def\csname PYG@tok@mf\endcsname{\def\PYG@tc##1{\textcolor[rgb]{0.13,0.50,0.31}{##1}}}
\expandafter\def\csname PYG@tok@gi\endcsname{\def\PYG@tc##1{\textcolor[rgb]{0.00,0.63,0.00}{##1}}}
\expandafter\def\csname PYG@tok@w\endcsname{\def\PYG@tc##1{\textcolor[rgb]{0.73,0.73,0.73}{##1}}}
\expandafter\def\csname PYG@tok@mh\endcsname{\def\PYG@tc##1{\textcolor[rgb]{0.13,0.50,0.31}{##1}}}
\expandafter\def\csname PYG@tok@nc\endcsname{\let\PYG@bf=\textbf\def\PYG@tc##1{\textcolor[rgb]{0.05,0.52,0.71}{##1}}}
\expandafter\def\csname PYG@tok@ni\endcsname{\let\PYG@bf=\textbf\def\PYG@tc##1{\textcolor[rgb]{0.84,0.33,0.22}{##1}}}
\expandafter\def\csname PYG@tok@nl\endcsname{\let\PYG@bf=\textbf\def\PYG@tc##1{\textcolor[rgb]{0.00,0.13,0.44}{##1}}}
\expandafter\def\csname PYG@tok@kp\endcsname{\def\PYG@tc##1{\textcolor[rgb]{0.00,0.44,0.13}{##1}}}
\expandafter\def\csname PYG@tok@ss\endcsname{\def\PYG@tc##1{\textcolor[rgb]{0.32,0.47,0.09}{##1}}}
\expandafter\def\csname PYG@tok@gs\endcsname{\let\PYG@bf=\textbf}

\def\PYGZbs{\char`\\}
\def\PYGZus{\char`\_}
\def\PYGZob{\char`\{}
\def\PYGZcb{\char`\}}
\def\PYGZca{\char`\^}
\def\PYGZam{\char`\&}
\def\PYGZlt{\char`\<}
\def\PYGZgt{\char`\>}
\def\PYGZsh{\char`\#}
\def\PYGZpc{\char`\%}
\def\PYGZdl{\char`\$}
\def\PYGZhy{\char`\-}
\def\PYGZsq{\char`\'}
\def\PYGZdq{\char`\"}
\def\PYGZti{\char`\~}
% for compatibility with earlier versions
\def\PYGZat{@}
\def\PYGZlb{[}
\def\PYGZrb{]}
\makeatother

\begin{document}

\maketitle
\tableofcontents
\phantomsection\label{index::doc}


Contents:


\chapter{SPHINX-DOC 安裝}
\label{_doc/sphinx-doc/index::doc}\label{_doc/sphinx-doc/index:sphinx-doc}\label{_doc/sphinx-doc/index:welcome-to-s-documentation}
目前對於一般使用者或系統開發者在記錄自已的學習心得的方式會隨著不同環境及狀況而所使用的工具有所不同,當在網路尚未普及到家家戶戶都接ADSL時,可能大都是寫在記事本、Microsft Office系列的文書軟體,但隨著網路的普及化加上全球資訊網出現,陸續出現BLOG、WIKI,到現在的雲端服務,如evernote、google雲端文件、Microsft Office系列等…,不論在桌上型電腦、筆記型電腦、平板、手機,在不同的作業系統都可以存取,但這些都注重於快速記事,如果要把所記錄的東西整合後出版到網路上公開,發佈成網頁、pdf、epub等格式,在出版這塊或許office是一般使用者最常用的,也可以發佈成PDF,而EPUB可以透過其他的軟體轉換,如果是發佈成網頁或許少數幾頁的文件是沒問題,大不了網頁內容長了點,但如果是一本書或使用手冊呢?將數十頁或數百頁的內容發佈成一個網頁,這就不太好觀看及使用了,這樣還需要花費更多的人力去建置網頁,檔案格式又不同了,當這個時後書本、使用手冊要改版時將會要修改多個版本的內容,再加上章節內容抽換時會出現文件格式變動的問題等…。上述的問題,在以建立書本、使用手冊為主,所要的需求如下:
\begin{itemize}
\item {} 
編輯一份文件,即可同時發佈成網頁、PDF、EPUB等格式

\item {} 
在文件改版,可容易抽換章節內容

\item {} 
可在不同的文件編輯器上寫作

\item {} 
便於多人共筆寫作

\end{itemize}

綜合以上的需求,SPHINX-DOC可以幫我們完成這個任務
SPHINX-DOC有別於其他的文書出版軟體,SPHINX-DOC起初是為了創建新的 \href{https://doc.python.org/3.5/index.html}{PYTHON使用者手冊} 而產生的,文件的內容是使用reStructuredText(reST)標記語言所編寫,reST是種跨平台及任何文字編號器都可以編輯,而且是個易讀的純文字標記語法,主要是透過Python中的Docutils元件將純文字轉換成不同的格式。


\section{測試環境}
\label{_doc/sphinx-doc/index:id1}\begin{itemize}
\item {} 
作業系統:Ubuntu 14.04

\item {} 
Python:3.4.0

\item {} 
SPHINX-DOC:1.2.2

\item {} 
Apache:2.4.7

\end{itemize}


\section{安裝SPHINX-DOC}
\label{_doc/sphinx-doc/index:id2}
本系統使用python3的版本做架設,在ubuntu有sphinx-doc套件可以直接安裝,而且是最新的版本1.2.2的版本,套件有分Python2和Python3的版本,兩種版本所產生出來的內容經比對過後,主要差異在因為中文使用Unicode,使用Python2的版本所產生出來的內容,如果有中文字,在前方會加一個''u''。
\begin{description}
\item[{\textbf{Python2和Python3中文差異}}] \leavevmode
\begin{Verbatim}[commandchars=\\\{\}]
python2:project = u\PYGZsq{}中文專案\PYGZsq{}
python3:project = \PYGZsq{}中文專案\PYGZsq{}
\end{Verbatim}

\item[{\textbf{安裝sphinx使用python3}}] \leavevmode
\begin{Verbatim}[commandchars=\\\{\}]
sudo apt\PYGZhy{}get install python3\PYGZhy{}sphinx
\end{Verbatim}

\item[{\textbf{安裝sphinx使用python2}}] \leavevmode
\begin{Verbatim}[commandchars=\\\{\}]
sudo apt\PYGZhy{}get install python\PYGZhy{}sphinx
\end{Verbatim}

\end{description}

安裝執行畫面


\section{建立SPHINX-DOC專案}
\label{_doc/sphinx-doc/index:id4}
建立專案執行畫面


\section{匯出HTML網站}
\label{_doc/sphinx-doc/index:html}
在Ubuntu下,不論是Python2或是Python3都是用「 \textbf{make html} 」指令建立HTML網站
\begin{description}
\item[{\textbf{匯出HTML指令}}] \leavevmode
\begin{Verbatim}[commandchars=\\\{\}]
make html
\end{Verbatim}

\item[{\textbf{執行畫面}}] \leavevmode
\begin{Verbatim}[commandchars=\\\{\}]
allen@uServer:\PYGZti{}/git/allen\PYGZdl{} make html
sphinx\PYGZhy{}build \PYGZhy{}b html \PYGZhy{}d \PYGZus{}build/doctrees   . \PYGZus{}build/html
Making output directory...
Running Sphinx v1.2.2
loading pickled environment... done
building [html]: targets for 6 source files that are out of date
updating environment: 0 added, 0 changed, 0 removed
looking for now\PYGZhy{}outdated files... none found
preparing documents... done
writing output... [100\PYGZpc{}] index
writing additional files... genindex search
copying static files... done
copying extra files... done
dumping search index... done
dumping object inventory... done
build succeeded.

Build finished. The HTML pages are in \PYGZus{}build/html.
\end{Verbatim}

\end{description}


\section{匯出PDF}
\label{_doc/sphinx-doc/index:pdf}

\section{匯出ePub}
\label{_doc/sphinx-doc/index:epub}
在Ubuntu下,不論是Python2或是Python3都是用「 \textbf{make epub} 」指令建立HTML網站
\begin{description}
\item[{\textbf{匯出HTML指令}}] \leavevmode
\begin{Verbatim}[commandchars=\\\{\}]
make epub
\end{Verbatim}

\item[{\textbf{執行畫面}}] \leavevmode
\begin{Verbatim}[commandchars=\\\{\}]
allen@uServer:\PYGZti{}/git/allen\PYGZdl{} make epub
sphinx\PYGZhy{}build \PYGZhy{}b epub \PYGZhy{}d \PYGZus{}build/doctrees   . \PYGZus{}build/epub
Making output directory...
Running Sphinx v1.2.2
loading pickled environment... done
building [epub]: targets for 6 source files that are out of date
updating environment: 0 added, 0 changed, 0 removed
looking for now\PYGZhy{}outdated files... none found
preparing documents... done
writing output... [100\PYGZpc{}] index
writing additional files... genindex search
copying static files... done
copying extra files... done
writing mimetype file...
writing META\PYGZhy{}INF/container.xml file...
writing content.opf file...
WARNING: unknown mimetype for \PYGZus{}static/test, ignoring
writing toc.ncx file...
writing sphinx.epub file...
build succeeded, 1 warning.

Build finished. The epub file is in \PYGZus{}build/epub.
\end{Verbatim}

\end{description}


\section{參考網站連結}
\label{_doc/sphinx-doc/index:id6}
\href{http://docutils.sourceforge.net/rst.html}{http://docutils.sourceforge.net/rst.html}
\href{http://docutils.sourceforge.net/docs/ref/rst/restructuredtext.html}{http://docutils.sourceforge.net/docs/ref/rst/restructuredtext.html}


\chapter{GIT 安裝}
\label{_doc/git/index:git}\label{_doc/git/index::doc}

\chapter{Ubuntu 安裝}
\label{_doc/ubuntu/index:ubuntu}\label{_doc/ubuntu/index::doc}

\chapter{test中文測試}
\label{_doc/test/index:test}\label{_doc/test/index::doc}\begin{itemize}
\item {} 
dtest

\item {} 
dkdkfe

\item {} 
1

\item {} 
2

\item {} 
3

\end{itemize}


\chapter{Indices and tables}
\label{index:indices-and-tables}\begin{itemize}
\item {} 
\emph{genindex}

\item {} 
\emph{modindex}

\item {} 
\emph{search}

\end{itemize}



\renewcommand{\indexname}{Index}
\printindex
\end{document}
